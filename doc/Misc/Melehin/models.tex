\section{Модели программных систем} % (fold)
\label{sec:models}

Одной из важнейших составляющих анализа программных систем является построение
модели. Без нее анализатор будет вынужден непосредственно оперировать с исходным
кодом, что влечет за собой усложнение процедур анализа и самого анализатора в
целом.

\subsection{Виды моделей программных систем} % (fold)

В зависимости от способа построения и назначения модели, они могу различаться по
структуре и сложности и обладать различными свойствами. Существуют следующие
виды моделей:

\begin{itemize}
    \item Структурные модели
    \item Поведенческие модели
    \item Гибридные модели
\end{itemize}

Структурные модели во основном используют информацию о синтаксической структуре
анализируемой программы, в то время как поведенческие - информацию о
динамической семантике. Гибридные модели используют оба этих подхода.

\subsubsection{Структурные модели} % (fold)
\begin{enumerate}
    \item Синтаксическое дерево

    Синтаксическое дерево является результатом разбора программы в
    соответствии с формальной грамматикой языка программирования. Вершины
    этого дерева соответствуют нетерминальным символам грамматики, а листья
    - терминальным.

    \item Абстрактное синтаксическое дерево

    Данная модель получается из обычного синтаксического дерева путем
    удаления нетерминальных вершин с одним потомком и замены части
    терминальных вершин их семантическими атрибутами.
\end{enumerate}

\subsubsection{Поведенческие модели} % (fold)
\begin{enumerate}
    \item Граф потока управления

    Граф потока управления представляет потоки управления программы в виде
    ориентированного графа. Вершинами графа являются операторы программы, а дуги
    отображают возможный ход исполнения программы и связывают между собой
    операторы, выполняемые друг за другом.

    \item Граф зависимостей по данным

    Граф зависимостей по данным отображает связь между конструкциями программы,
    зависимыми по используемым данным. Дуги графа соединяют узлы, формирующие
    данные, и узлы, использующие эти данные.

    \item Граф программных зависимостей

    Данная модель объединяет в себе особенности графа потока управления и графа
    зависимости по данным. В графе программных зависимостей присутствуют дуги
    двух типов: информационные дуги отображают зависимости по данным, а
    дуги управления соединяют последовательно выполняемые конструкции.

    \item Представление в виде SSA

    Однократное статическое присваивание (static single assignment) -
    промежуточное представление программы, которое обладает следующими
    свойствами:
        \begin{itemize}
            \item Всем переменным значение может присваиваться только один раз.
            \item Вводится специальный оператор $\phi$-функция, который объединяет
            разные версии локальных переменных.
            \item Все операторы программы представляются в трехоперандной форме.
        \end{itemize}
\end{enumerate}

\subsubsection{Гибридные модели} % (fold)
\begin{enumerate}
    \item Абстрактный семантический граф

    Данная модель является расширением абстрактного синтаксического дерева путем
    добавления дуг, отражающих некоторые семантически свойства программы,
    например, такие дуги могут связывать определение и использование переменной
    или определение функции и ее вызов.
\end{enumerate}
