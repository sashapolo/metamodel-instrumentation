\documentclass{beamer}

\usepackage[T2A]{fontenc}
\usepackage[utf8]{inputenc}
\usepackage[russian]{babel}

\hypersetup {
    unicode = true
}

\usetheme{Madrid}
\usecolortheme{whale}

\title[Разработка фреймворка]
{Разработка фреймворка для анализа и верификации программных систем}
\author[А.М. Полоцев]{
    А.М. Полоцев гр. 63501/13\\
    Руководитель: Ицыксон В.М.\\
    Аттестация №1
}
\date[18.10.2013]{}

\begin{document}

\frame{\titlepage}

\begin{frame}{Описание задачи}
    При проведении анализа и верификации программных систем,
    построения метрик и т.д., часто решаются похожие задачи:
    \begin{itemize}
        \item построение CFG
        \item построение AST
        \item вычисление различных метрик
        \item восстановление исходного кода по модели
        \item инструментирование
        \item ...
    \end{itemize}
    Для решения данных задач существуют специализированные фреймворки.
    Необходимо провести анализ существующих решений и, в случае неудовлетворения
    этих решений поставленным требованиям, разработать собственное.
\end{frame}

\begin{frame}{План работы}
    \begin{itemize}
        \item[\checkmark] формулирование требований к фреймворку
        \item[\checkmark] анализ существующих фреймворков анализа программ
        \item постановка задачи магистерского исследования
        \item анализ требований к фреймворку
        \item проектирование архитектуры фреймворка
        \item проектирование API фреймворка
        \item реализация
        \item тестирование
        \item написание пояснительной записки
    \end{itemize}
\end{frame}

\begin{frame}{Что сделано}
    Проведен анализ уже существующих фреймворков:
    \begin{itemize}
        \item Moose
        \item Smile
        \item IRE
        \item LLVM
    \end{itemize}
    На данный момент производится анализ этих средств и их проверка на
    соответствие предъявляемым требованиям.
\end{frame}

\begin{frame}{Спасибо за внимание}
\center{\resizebox{60pt}{80pt}{?}}
\end{frame}

\end{document}
