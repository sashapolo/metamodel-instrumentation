\documentclass[%
  a5paper,
  subf,
  href,
  master,
  dotsinheaders
]{csse-fcs}

\usepackage[T2A]{fontenc}
\usepackage[utf8]{inputenc}
\usepackage[english,russian]{babel}
\usepackage{enumitem}

\begin{document}

\pagestyle{empty}
\begin{center}
    Санкт-Петербургский государственный политехнический университет\\
    Институт информационных технологий и управления \\
    Кафедра компьютерных систем и программных технологий \\
\end{center}
\begin{flushright}
    \MakeUppercase{Утверждаю} \\
    Зав. кафедрой\\
    \rule{5em}{1sp}, В.Ф.~Мелехин\\
    <<\rule{2em}{1sp}>> \rule{7em}{1sp} \the\year~г.
\end{flushright}
\begin{center}
    \bf
    \MakeUppercase{Задание} \\
    на магистерскую диссертацию \\
    студента Половцева Александра Михайловича
\end{center}

\begin{enumerate}
    \item \textbf{Тема работы:} Инструментальная среда для анализа программных
    систем.
    \item \textbf{Срок сдачи студентом законченной работы:} 05.06.2014
    \item \textbf{Содержание расчетно"=пояснительной записки:}
    \begin{enumerate}[label=\arabic{*})]
        \item Анализ подходов и средств построения инструментальных средств
        \item Постановка задач и требований к среде
        \item Проектирование архитектуры инструментальной среды
        \item Реализация инструментальной среды
        \item Тестирование разработанной программной системы
    \end{enumerate}
    \item \textbf{Дата выдачи задания:} 12.09.2013
\end{enumerate}

\begin{flushright}
    Руководитель: \rule{7em}{1sp} (В.М.~Ицыксон) \\
    Задание принял к исполнению <<12>> сентября 2013: \\
    \rule{7em}{1sp} (А.М.~Половцев) \\
\end{flushright}

\end{document}
