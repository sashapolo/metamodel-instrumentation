% !TEX root = thesis.tex
%%%%%%%%%%%%%%%%%%%%%%%%%%%%%%%%%%%%%%%%%%%%%%%%%%%%%%%%%%%%%%%%%%%%%%%%%%%%%%%%
\intro
%%%%%%%%%%%%%%%%%%%%%%%%%%%%%%%%%%%%%%%%%%%%%%%%%%%%%%%%%%%%%%%%%%%%%%%%%%%%%%%%

\nomenclature{ПО}{Программное обеспечение}

В данной работе рассматривается подход к автоматизации процесса проведения
анализа и верификации программных систем с целью повышения характеристик
качества.

С развитием вычислительных систем и ростом в них доли программной составляющей,
сложность разрабатываемых программ постоянно возрастает. Также, вследствие
большой конкуренции на рынке программного обеспечения, постоянно снижаются сроки
разработки новых версий ПО. Эти факторы неизбежно ведут к снижению качества
выпускаемых продуктов.

Падение уровня качества является проблемой, особенно если программное
обеспечение задействовано в критически важных сферах человеческой деятельности,
например медицине и космонавтике, так как наличие в них ошибок ведет к большому
материальному ущербу и даже человеческим жертвам. Поэтому задача повышения
качества является одной из самых актуальных в сфере информационных технологий.

Одними из способов повышения качества программ являются статический анализ и
формальные методы, которые часто реализуются в виде инструментальных
средств. При разработке данных средств часто решаются похожие задачи, такие
как:
\begin{itemize}
    \item Построение моделей программы, например, абстрактного синтаксического
    дерева, графа потока управления, графа программных зависимостей и т.д.
    (модели программ рассмотрены в подразделе~\ref{sec:system_models})
    \item Построение различных метрик программного кода
    \item Реинжиниринг программного обеспечения (оптимизация, рефакторинг и т.п.)
    \item Визуализация свойств программной системы
    \item и т.п.
\end{itemize}

Более подробно методы обеспечения качества рассмотрены в
подразделе~\ref{sec:quality_methods}.

Обычно эти задачи решаются вручную каждый раз при создании анализаторов или
проведения верификации программы. В данной работе предлагается способ
автоматизации решения этих задач на основе использования представлений
программы, не зависящих от языка написания ее исходного кода, называемых
метамоделями.
