% !TEX root = thesis.tex
%%%%%%%%%%%%%%%%%%%%%%%%%%%%%%%%%%%%%%%%%%%%%%%%%%%%%%%%%%%%%%%%%%%%%%%%%%%%%%%%
\conclusion
%%%%%%%%%%%%%%%%%%%%%%%%%%%%%%%%%%%%%%%%%%%%%%%%%%%%%%%%%%%%%%%%%%%%%%%%%%%%%%%%

В результате выполнения диссертации была разработана инструментальная среда,
позволяющая автоматизировать процедуры анализа и верификации. Для решения данной
задачи был предложен подход с использованием языконезависимой метамодели. Таким
образом, цель диссертации была успешно достигнута.

В ходе работы были рассмотрены методы повышения качества и используемые в них
модели ПО (разд.~\ref{chap:analisys}). После обзора способов абстрагирования от
языка программирования для унификации процедур анализа был проведен обзор
существующих средств, использующих метамоделирование. На основе этого обзора
было принято решение о целесообразности разработки собственного средства.

На основе анализа существующих средств и требований, поставленных в
разд.~\ref{chap:task},  была предложена архитектура инструментальной среды,
которая была разделена на три фрагмента - преобразователи, метамодель и
процедуры анализа (разд.~\ref{chap:architeture}). Для каждой составляющей были
предложены варианты реализации.

В разд.~\ref{chap:realisation} была описана реализация разработанной
архитектуры. При реализации использовались приемы и паттерны объектно-
ориентированного программирования, использовался широкий круг библиотек для
решения задач синтаксического разбора, сериализации и визуализации.

На заключительном этапе разработки (разд.~\ref{chap:testing}) было проведено
функциональное тестирование среды, показавшее соответствие разработанной системы
поставленным требованиям.
