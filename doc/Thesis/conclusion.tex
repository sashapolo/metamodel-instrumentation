% !TEX root = thesis.tex
%%%%%%%%%%%%%%%%%%%%%%%%%%%%%%%%%%%%%%%%%%%%%%%%%%%%%%%%%%%%%%%%%%%%%%%%%%%%%%%%
\conclusion
%%%%%%%%%%%%%%%%%%%%%%%%%%%%%%%%%%%%%%%%%%%%%%%%%%%%%%%%%%%%%%%%%%%%%%%%%%%%%%%%

В результате выполнения диссертации была разработана инструментальная среда,
позволяющая автоматизировать процедуры анализа и верификации. Для решения данной
задачи был предложен подход с использованием языконезависимой метамодели.

В ходе работы были рассмотрены методы повышения качества и используемые в них
модели ПО (раздел~\ref{chap:analisys}). После обзора способов абстрагирования от
языка программирования для унификации процедур анализа был проведен обзор
существующих средств, использующих метамоделирование. На основе этого обзора
было принято решение о целесообразности разработки собственного средства.

На основе анализа существующих средств и требований, поставленных в
разделе~\ref{chap:task}, была предложена архитектура инструментальной среды,
которая была разделена на три фрагмента - преобразователи, метамодель и
процедуры анализа (раздел~\ref{chap:architeture}). Для каждой составляющей были
предложены варианты реализации.

В разделе~\ref{chap:realisation} была описана реализация разработанной
архитектуры. При реализации использовались приемы и паттерны объектно-
ориентированного программирования, использовался широкий круг библиотек для
решения задач синтаксического разбора, сериализации и визуализации.

На заключительном этапе разработки (раздел~\ref{chap:testing}) было проведено
функциональное тестирование среды, показавшее соответствие разработанной системы
поставленным требованиям.

Разработанную инструментальную среду можно применять для визуализации свойств
программных систем при проведении различных неформальных методов повышения
качества, например, аудита. Функция подсчета метрик позволяет оценивать
характеристики проекта на поздних этапах его жизненного цикла, а визуализацию
графа потока управления можно применять для отладки каких-либо процедур
анализируемой системы.

Библиотека с метамоделью может быть использована как промежуточное представление
в формальных методах анализа. Так как метамодель является независимой от языка
программирования, на котором написана анализируемая программная система,
разработанные процедуры анализа можно применять для широкого набора систем.

Дальнейшим развитием проекта является разработка новых видов визуализации,
процедур извлечения новых видов моделей и различных процедур анализа на
основе этих моделей. Также планируется добавление преобразователей для других
популярных языков программирования.
