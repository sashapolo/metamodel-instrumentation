% !TEX root = thesis.tex
%%%%%%%%%%%%%%%%%%%%%%%%%%%%%%%%%%%%%%%%%%%%%%%%%%%%%%%%%%%%%%%%%%%%%%%%%%%%%%%%
\chapter{Постановка задачи}
%%%%%%%%%%%%%%%%%%%%%%%%%%%%%%%%%%%%%%%%%%%%%%%%%%%%%%%%%%%%%%%%%%%%%%%%%%%%%%%%
Как было сказано в разделе~\ref{sec:system_models}, инструменты для проведения
статического анализа и верификации используют различные модели программ для
облегчения процедур анализа. Однако, данные модели зависят от языка, на котором
написана система, а, следовательно, при таком подходе невозможно обобщить
разработанные алгоритмы.

К тому же проблема зависимости процедур анализа программного обеспечения от
исходного текста остро стоит не только при проведении верификации, но и при
решении задач реинжиниринга и оптимизации, а именно:

\begin{itemize}
    \item Построение метрик
    \item Визуализация свойств системы
    \item Поиск клонов
    \item Анализ истории проекта
\end{itemize}

Таким образом, ставится задача разработки среды, которая бы позволила
абстрагировать алгоритмы анализа и реинжиниринга от языка описания системы. Это
позволит применять разработанные средства для гораздо более широкого круга
систем, автоматизировав процесс извлечения модели.

Исходя из вышеперечисленного, разрабатываемое средство должно отвечать
следующим сценариям использования:

\begin{enumerate}
    \item Библиотека для извлечения моделей, необходимых для написания процедур
    статического анализа и верификации
    \item Инструментальная среда для визуализации свойств полученных моделей
\end{enumerate}

\subsection{}
