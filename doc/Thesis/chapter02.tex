% !TEX root = thesis.tex
%%%%%%%%%%%%%%%%%%%%%%%%%%%%%%%%%%%%%%%%%%%%%%%%%%%%%%%%%%%%%%%%%%%%%%%%%%%%%%%
\chapter{Постановка задачи}
\label{sec:task}
%%%%%%%%%%%%%%%%%%%%%%%%%%%%%%%%%%%%%%%%%%%%%%%%%%%%%%%%%%%%%%%%%%%%%%%%%%%%%%%%
Как было сказано в разделе~\ref{sec:system_models}, инструменты для проведения
статического анализа и верификации используют различные модели программ для
облегчения процедур анализа. Однако, данные модели зависят от языка, на котором
написана система, а, следовательно, при таком подходе невозможно обобщить
разработанные алгоритмы.

К тому же проблема зависимости процедур анализа программного обеспечения от
исходного текста остро стоит не только при проведении верификации, но и при
решении задач реинжиниринга и оптимизации, а именно:

\begin{itemize}
    \item Построение метрик
    \item Визуализация свойств системы
    \item Поиск клонов
    \item Анализ истории проекта
\end{itemize}

%%%%%%%%%%%%%%%%%%%%%%%%%%%%%%%%%%%%%%%%%%%%%%%%%%%%%%%%%%%%%%%%%%%%%%%%%%%%%%%%
\section{Формулирование требований к инструментальной среде}
%%%%%%%%%%%%%%%%%%%%%%%%%%%%%%%%%%%%%%%%%%%%%%%%%%%%%%%%%%%%%%%%%%%%%%%%%%%%%%%%

\todo{Сделать требования более понятными. Расписать про все возможности системы. Требования к API для работы с метамоделью}

Таким образом, исходя из описанных проблем, ставится задача разработки среды,
которая бы позволила абстрагировать алгоритмы анализа и реинжиниринга от языка
описания системы. Это позволит применять разработанные средства для гораздо
более широкого круга систем, автоматизировав процесс извлечения модели.

Разрабатываемое средство должно предоставлять следующие возможности:

\begin{enumerate}
    \item Извлечение моделей, необходимых для написания процедур статического
    анализа и верификации
    \item Визуализация следующих моделей и свойств:
        \begin{itemize}
            \item CFG
            \item AST
            \item Диаграмма классов
            \item Подсчет метрик
        \end{itemize}
\end{enumerate}

Ядром всей системы и средством абстракции является метамодель, к ней
предъявляются следующие требования:

\begin{enumerate}
    \item Независимость от языка описания системы
    \item Расширяемость - возможность добавлять новые сущности по мере
    необходимости
    \item Простота в использовании
\end{enumerate}

%%%%%%%%%%%%%%%%%%%%%%%%%%%%%%%%%%%%%%%%%%%%%%%%%%%%%%%%%%%%%%%%%%%%%%%%%%%%%%%%
\section{Выбор пути решения}
\label{sec:tasks}
%%%%%%%%%%%%%%%%%%%%%%%%%%%%%%%%%%%%%%%%%%%%%%%%%%%%%%%%%%%%%%%%%%%%%%%%%%%%%%%%
Как видно из обзора в п.~\ref{sec:analysis}, ни одно из рассмотренных средств не
отвечает поставленным требованиям, поэтому было принято решение о написании
собственной инструментальной среды. При этом предполагается решить следующие
задачи:

\begin{enumerate}
    \item Проектирование структуры метамодели, отвечающей всем поставленным
    требованиям
    \item Проектирование графической инструментальной среды, поддерживающей все
    необходимые виды визуализации
    \item Реализация и тестирование программной системы
    \item Анализ полученных результатов
\end{enumerate}

%%%%%%%%%%%%%%%%%%%%%%%%%%%%%%%%%%%%%%%%%%%%%%%%%%%%%%%%%%%%%%%%%%%%%%%%%%%%%%%%
\section{Вывод}
%%%%%%%%%%%%%%%%%%%%%%%%%%%%%%%%%%%%%%%%%%%%%%%%%%%%%%%%%%%%%%%%%%%%%%%%%%%%%%%%
В результате успешного выполнения задач, приведенных в п.п.\ref{sec:tasks},
необходимо получить инструментальную среду, поддерживающую несколько видов
визуализации свойств программных систем, оставаясь при этот независимой от
языка описания этих систем.

