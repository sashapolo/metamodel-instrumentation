% !TEX root = thesis.tex
%%%%%%%%%%%%%%%%%%%%%%%%%%%%%%%%%%%%%%%%%%%%%%%%%%%%%%%%%%%%%%%%%%%%%%%%%%%%%%%
\chapter{Постановка задачи}
\label{chap:task}
%%%%%%%%%%%%%%%%%%%%%%%%%%%%%%%%%%%%%%%%%%%%%%%%%%%%%%%%%%%%%%%%%%%%%%%%%%%%%%%%
Как было сказано в разделе~\ref{sec:system_models}, инструменты для проведения
статического анализа и верификации используют различные модели программ для
облегчения процедур анализа. Однако данные модели зависят от языка, на котором
написана система, а, следовательно, при таком подходе невозможно обобщить
разработанные алгоритмы.

К тому же проблема зависимости процедур анализа программного обеспечения от
исходного текста остро стоит не только при проведении верификации, но и при
решении задач реинжиниринга и оптимизации, а именно:

\begin{itemize}
    \item Построение метрик
    \item Визуализация свойств системы
    \item Поиск клонов
    \item Анализ истории проекта
\end{itemize}

%%%%%%%%%%%%%%%%%%%%%%%%%%%%%%%%%%%%%%%%%%%%%%%%%%%%%%%%%%%%%%%%%%%%%%%%%%%%%%%%
\section{Формулирование требований к инструментальной среде}
%%%%%%%%%%%%%%%%%%%%%%%%%%%%%%%%%%%%%%%%%%%%%%%%%%%%%%%%%%%%%%%%%%%%%%%%%%%%%%%%

Таким образом, исходя из описанных проблем, ставится задача разработки среды,
которая бы позволила абстрагировать алгоритмы анализа и реинжиниринга от языка
описания системы. Это позволит применять разработанные средства для гораздо
более широкого круга систем, автоматизировав процесс извлечения модели.

Ядром всей системы и средством абстракции является метамодель, которая должна
обладать следующими свойствами:

\nomenclature{API}{Application programming interface, интерфейс прикладного
программирования}

\begin{enumerate}
    \item Независимость от языка описания системы - метамодель должна
    поддерживать несколько парадигм программирования (как минимум, структурную и
    объектно-ориентированную) и обладать достаточной полнотой для
    описания специфических конструкций конкретного языка.
    \item Расширяемость - возможность добавлять новые сущности по мере
    необходимости.
    \item Простота в использовании - метамодель должна предоставлять удобный API
    для облегчения обхода структуры модели и реализации различных процедур
    анализа ПО. Исходя из возможного изменения структуры метамодели API должен
    предусматривать возможность работы с добавленными узлами.
    \item Полнота - метамодель должна содержать достаточное количество
    информации для извлечения различных моделей, описанных в
    подразделе~\ref{sec:system_models}, а также других свойств анализируемой
    системы (например, метрик).
\end{enumerate}

Для демонстрации возможностей метамодели необходимо разработать инструментальную
среду, позволяющую выполнять следующие действия:

\begin{enumerate}
    \item Визуализация извлеченных моделей
    \item Подсчет метрик
    \item Визуализация дополнительных свойств системы (например, диаграммы
    классов для объектно-ориентированной системы)
    \item Импортирование систем на языках Java и C (являющимися наиболее
    популярными языками программирования и реализующие две разные парадигмы)
\end{enumerate}

%%%%%%%%%%%%%%%%%%%%%%%%%%%%%%%%%%%%%%%%%%%%%%%%%%%%%%%%%%%%%%%%%%%%%%%%%%%%%%%%
\section{Решаемые задачи}
\label{sec:tasks}
%%%%%%%%%%%%%%%%%%%%%%%%%%%%%%%%%%%%%%%%%%%%%%%%%%%%%%%%%%%%%%%%%%%%%%%%%%%%%%%%

Как видно из обзора в подразделе~\ref{sec:analysis}, ни одно из рассмотренных
средств не отвечает поставленным требованиям, поэтому было принято решение о
написании собственной инструментальной среды. При этом предполагается решить
следующие задачи:

\begin{enumerate}
    \item Проектирование структуры метамодели, отвечающей всем поставленным
    требованиям
    \item Проектирование графической инструментальной среды, поддерживающей все
    необходимые виды визуализации
    \item Реализация и тестирование программной системы
\end{enumerate}

В результате успешного выполнения данных задач, необходимо получить метамодель,
позволяющую описывать программные системы, абстрагируясь от использованного
языка программирования, а также инструментальную среду для визуализации ее
различных свойств.

%%%%%%%%%%%%%%%%%%%%%%%%%%%%%%%%%%%%%%%%%%%%%%%%%%%%%%%%%%%%%%%%%%%%%%%%%%%%%%%%
\section{Вывод}
%%%%%%%%%%%%%%%%%%%%%%%%%%%%%%%%%%%%%%%%%%%%%%%%%%%%%%%%%%%%%%%%%%%%%%%%%%%%%%%%

В данном разделе были сформулированы требования к разрабатываемой программной
системе, а также поставлены задачи, которые необходимо решить для достижения
цели диссертации - разработки способа автоматизации процедур анализа и
визуализации свойств программных систем.

