\chapter{Анализ подходов и средств инструментирования программ}

\section{Методы повышения качества}
\label{sec:quality_methods}

Существует две группы подходов по обеспечению качества программного обеспечения:

\subsection{Подходы, основанные на синтезе ПО}

Данная группа подходов использует различные формализации во время проектирования
системы, таким образом позволяя избежать ошибок на более поздних этапах
разработки.

Данные формализации включают в себя~\cite{itsykson}:

\begin{itemize}
    \item формальные спецификации
    \item формальные и неформальные описания различных аспектов программной
    системы
    \item архитектурные шаблоны и стили
    \item паттерны проектирования
    \item генераторы шаблонов программ
    \item генераторы программ
    \item контрактное программирование
    \item аннотирование программ
    \item верификация моделей программ с использованием частичных спецификаций
    \item использование моделей предметной области для автоматизации
    тестирования программ
\end{itemize}

\section{Постановка требований к инструментальной среде}
\section{Анализ существующих решений}
